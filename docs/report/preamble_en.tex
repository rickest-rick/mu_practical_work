%------- Commands -----------------------------------------------------

\newcommand{\N}{\mathbb{N}}
\newcommand{\Z}{\mathbb{Z}}
\newcommand{\Q}{\mathbb{Q}}
\newcommand{\R}{\mathbb{R}}
\newcommand{\C}{\mathbb{C}}
\newcommand{\F}{\mathbb{F}}

\newcommand{\Aufg}{\stepcounter{Nummer}
	\textbf{Aufgabe \arabic{Nummer}:}}
\newcommand{\Lsg}{\textbf{L\"osungskommentar \arabic{Nummer}:}}
%------- Sprache --------------------------------------------------

% deutsche Silbentrennung
\usepackage[english]{babel}
\let\latinencoding\relax



% eigene, zusätzliche Silbentrennung
\usepackage{hyphsubst} %Manuelle Sil\-ben\-tren\-nung



% Umlaute#
\usepackage[utf8]{inputenc}
 
  
  
\usepackage{amsmath}
\usepackage{amsfonts}
\usepackage{amssymb}
\usepackage{verbatim}  
  

%------- Formatierung --------------------------------------------------
  
% Seitenränder
\usepackage{geometry}
\geometry{a4paper, top= 25mm, bottom=20mm, left = 25mm, right = 25mm, footskip= 1cm}



% Querformat auf einzelner Seite \begin{landscape}

%\usepackage{lscape}


% Zeilenabstand (Unterscheidet sich evtl von den Standard 1.5)
%\usepackage{setspace}
%\linespread{1.5}


% no indent fürs Dokument.
\setlength\parindent{0pt}



% Seitenzahlen, hakt sich mit Kopfzeilen
% \pagestyle{plain}
% Späterer Beginn der Seitennummerierung: Kombinationen aus
%\thispagestyle{empty} und \setcounter{page}{1}



% Kopfzeilen



%Größen von Chapter usw ändern
%\makeatletter
%\renewcommand*{\size@chapter}{\Large}
%\renewcommand*{\size@section}{\large}
%\makeatother


%Kapitel ohne Nummer, aber im Inhaltsverzeichnis:
%\chapter*{Literaturverzeichnis}\addcontentsline{toc}{chapter}{Literaturverzeichnis}



%Fußnote \footnote{\label{foot:1}Korrekt gelesen!}

%------- Grafiken --------------------------------------------------

\usepackage{graphicx}
%\includegraphics[width=0.5\textwidth]{images/img1} oder scale=1





%------- Titelseite --------------------------------------------------

\begin{comment}

\begin{titlepage}
    \begin{center}
    \ \\ \vspace{30mm}
    \large \textbf{\textsf{Theoriebasierte Praxisreflexion}} \\
    \vspace{5mm}
    \LARGE\textbf{\textsf{Humor im Unterricht}}\\
    \vspace{1cm}
    \normalsize
    \today \\
    \vspace{9cm}
    \end{center}
 \normalsize{Tony Prange\\Boeselagerstraße 69a\\48163 Münster\\Matrikelnummer: 398873\\Email: tony.prange@wwu.de\\Abgabetermin 1. September 2016}
\end{titlepage}

\newpage  \tableofcontents \thispagestyle{empty} 

\end{comment}