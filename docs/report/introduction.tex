\section{Introduction}
A large amount of data is available through the networking of our digital devices, which shape our everyday lives. The use of such data and the associated information is a core element of today's digital society. Programs that use such data to generate as much information as possible play an important role at assisting us and improving our daily lives. These programs are used in great variety of real world contexts, not only at the level of targeted advertising, but also in improving our medical care and our behaviors. For example, many people are already optimizing their personal lifestyle with the assistance of digital devices like smartphones and fitness trackers. Another application is the establishment of a healthy sleep rhythm by the recognition of sleep phases. 

In the context of health care, this technology is of particular interest because on the one hand it can be used in a preventive manner. For example, people can be informed about their personal habits and sensitized to their personal risks which arise with their specific lifestyle. This might encourage them to a healthier lifestyle, which in turn can be supported by technology again. On the other hand there is the medical perspective, where it is possible to check how healthy a person lives in order to combat diseases more effectively. In conclusion, deriving activities from various sensor data is important for many companies or health organizations, but also for the end user.

The present report was written during a practical course, led by Prof. Dr. Xiaoyi Jiang and Sören Klemm, with the topic \gl Behavioral context recognition in-the-wild from mobile sensors\gr{} accompanying the lecture Pattern Recognition. The aim of the project was to perform multi-label classification on the basis of the ExtraSensory data set. Essentially, collected sensor data should be assigned to activities, in an additional task the users should be recognized by their data. 

The particular challenge of the project was that firstly the methodology was not specified and secondly multi-label classification was to be carried out with a dataset containing many missing labels. 