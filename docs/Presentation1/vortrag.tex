%!TEX TS-program = pdflatex
\documentclass[10pt,%
	wide,%
	xcolor={x11names},%
	hyperref={colorlinks},%
	pantone312,%
	handout,%
	]{beamer}
\input{configBeamer.tex}
\author{Daniel Beckman, Thomas Poschadel, Tony Prange, Joschka Strüber}
\title{Behavioral Context Recognition}
\subtitle{Praktikum Mustererkennung}
\date{\today}

\begin{document}
\setbeamertemplate{section in toc}[sections numbered]

\begin{frame}[plain]
  \maketitle
\end{frame}

\begin{frame}[t]{Aufbau}
\tableofcontents[hidesubsections, hideothersubsections]
\end{frame}

\begin{frame}
	\frametitle{Saeed et al.: [...] multi-stream temporal convolutional networks}
	\begin{center}
		\includegraphics[width=0.85\textwidth]{img/multi-modal-network.png}
	\end{center}
\end{frame}

\begin{frame}
	\frametitle{Saeed et al.: [...] multi-stream temporal convolutional networks}
	\begin{itemize}
		\item \textbf{Learning behavioral context recognition with multi-stream temporal convolutional networks} (Aaqib Saeed, Tanir Ozcelebi, Stojan Trajanovski, Johan Lukkien), 08/2018 \cite{saeed2018learning}
		\item Aufteilung der Daten bezüglich ihrer Art (Audio, Acc-Sensor Daten)
		\item Netze für die Modalitäten werden in einem weiteren Netz zusammengeführt
		\item \emph{modal-specific networks} nutzen \emph{depthwise separable convolution}
		\item Keine Verbesserung gegenüber standardmäßiger Convolution, aber weniger Parameter und damit effizienter
	\end{itemize}
\end{frame}

\end{document}
