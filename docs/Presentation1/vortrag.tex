%!TEX TS-program = pdflatex
\documentclass[10pt,%
	wide,%
	xcolor={x11names},%
	hyperref={colorlinks},%
	pantone312,%
	handout,%
	]{beamer}
\input{configBeamer.tex}
\author{Daniel Beckman, Thomas Poschadel, Tony Prange, Joschka Strüber}
\title{Behavioral Context Recognition}
\subtitle{Praktikum Mustererkennung}
\date{\today}

\begin{document}
\setbeamertemplate{section in toc}[sections numbered]

\begin{frame}[plain]
  \maketitle
\end{frame}

\begin{frame}[t]{Aufbau}
\tableofcontents[hidesubsections, hideothersubsections]
\end{frame}

\section{Multi-stream Temporal Convolutional Networks}

\begin{frame}
\frametitle{Saeed et al.: [...] multi-stream temporal convolutional networks}
\begin{center}
	\includegraphics[width=0.85\textwidth]{img/multi-modal-network.png}
\end{center}
\end{frame}

\begin{frame}
\frametitle{Saeed et al.: [...] multi-stream temporal convolutional networks}
\begin{itemize}
\item \textbf{Learning behavioral context recognition with multi-stream temporal convolutional networks} \cite{saeed2018learning}
\item Aufteilung der Daten bezüglich ihrer Art (Audio, Acc-Sensor Daten)
\item Netze für die Modalitäten werden in einem weiteren Netz zusammengeführt
\item \emph{modal-specific networks} nutzen \emph{depthwise separable convolution}
\item Keine Verbesserung gegenüber standardmäßiger Convolution, aber weniger Parameter und damit effizienter
\end{itemize}
\end{frame}


\section{Adversarial Autoencoder (AAE)}

\begin{frame}[t]{Probleme bei den Sensordaten}
	\begin{itemize}
		\item fehlende Sensordaten
		\begin{itemize}
			\item keine WLAN-Verbindung
			\item Entscheidung, keine Smartwatch zu tragen
			\item Verbot, Bewegungsdaten auszuwerten
		\end{itemize}
		\item unbalancierte Datenlage
		\begin{itemize}
			\item \enquote{at the beach} seltenere Aktivität als \enquote{standing}
		\end{itemize}
	\end{itemize}

\textbf{Lösungsansatz:}
\begin{itemize}
	\item Adversarial Autoencoder \cite{sol18}
	\begin{itemize}
		\item Rekonstruktion fehlender Sensordaten
		\item Generierung von realistischen synthetischen Daten
	\end{itemize}
\end{itemize}
\end{frame}

\begin{frame}[t]{Adversarial Autoencoder}
	\begin{figure}
		\centering
		\includegraphics[scale=4.0]{img/aae}
		\caption{Framework für Kontextklassifizierung mit fehlenden Sensordaten \cite{sol18}}
	\end{figure}
\end{frame}

\begin{frame}[t]{Evaluierung}
	\begin{itemize}
		\item Klassifizierungsergebnisse vergleichbar mit leichteren Standardtechniken (Mean, Fill-1, PCA) $\Rightarrow$ liegt vermutlich an zu geringem Umfang der fehlenden Daten
		\item eingebautes GAN kann komplette realistische Datensätze synthetisieren
		\begin{itemize}
			\item Training nur auf synthetischen Daten liefert fast so gute Ergebnisse wie Training auf echten Daten (0.715 zu 0.752)
			\item Ergänzung von Daten für seltene Label könnte Klassifikation robuster machen
		\end{itemize}
		\item Möglichkeit sich mit interessanten Techniken (GAN, Autoencoder) auseinanderzusetzen
	\end{itemize}
\end{frame}

\section{Rekurrierende Neuronale Netze}

\begin{frame}[t]{Inoue, M., Inoue, S. \& Nishida  : Deep recurrent neural network for mobile human activity recognition with high throughput}
\begin{itemize}
	\item Nutzung von Rekurrierenden Neuronalen Netzen (RNN)
	\item Datensatz: von HASC (hasc.jp) mit Aktivitäten wie ''stay'', ''walk'', ''jog'', ''skip'', ''stair up'', ''stair down'
\end{itemize}
\begin{center}
	\includegraphics[width=0.5\textwidth]{img/Inoue1}

	\footnote{image source: https://de.wikipedia.org/wiki/Rekurrentes\_neuronales\_Netz\#/media/File:Neuronal-Networks-Feedback.png}
\end{center}

\end{frame}

\begin{frame}[t]{Inoue, M., Inoue, S. \& Nishida  : Deep recurrent neural network for mobile human activity recognition with high throughput}
Vorzüge
\begin{itemize}
\item keine Features von vornherein benötigt
\item wenige Knoten im neuronalen Netz
\item gute Performance 
	\begin{itemize} 
	\item vorausgesetzt, die Feature-Zeit wird mit einberechnet
	\item vorausgesetzt, die Datasets sind vergleichbar
	\end{itemize}
\end{itemize}

\begin{center}
\includegraphics[width=0.4\textwidth]{img/Inoue2}
\includegraphics[width=0.4\textwidth]{img/Inoue3}
\end{center}

\end{frame}

\section*{Quellen}
\begin{frame}[allowframebreaks,t]{\secname}
\printbibliography
\end{frame}

\end{document}
