%!TEX TS-program = pdflatex
\documentclass[10pt,%
	wide,%
	xcolor={x11names},%
	hyperref={colorlinks},%
	pantone312,%
	handout,%
	]{beamer}
\input{configBeamer.tex}
\author{Daniel Beckmann, Thomas Poschadel, Tony Prange, Joschka Strüber}
\title{Behavioral Context Recognition}
\subtitle{Praktikum Mustererkennung II}
\date{\today}

\begin{document}
\setbeamertemplate{section in toc}[sections numbered]

\begin{frame}[plain]
  \maketitle
\end{frame}

\begin{frame}[t]{Aufbau}
\tableofcontents[hidesubsections, hideothersubsections]
\end{frame}

\section{Was wir bisher gemacht haben}

\begin{frame}[t]{Kennenlernen des Datensatzes und Benutzererkennung}

\end{frame}

\begin{frame}[t]{Klassifizierung mit XGBoost}
	\begin{itemize}
		\item Bibliothek für GPU-unterstützte und verteilte Berechnung von \emph{Gradient Boosted Trees}
	\end{itemize}
	Vorteile:
	\begin{itemize}
		\item liefert gute Ergebnisse für tabulare Daten
		\item Scikit-learn API vorhanden $\rightarrow$ Verwendung der Scikit-learn Infrastruktur gut möglich, insbesondere \emph{OneVsRestClassifier}
		\item gute Interpretierbarkeit $\rightarrow$ 
	\end{itemize}
\end{frame}

\begin{frame}[t]{Hyperparametertuning mit randomisierter Suche und Bayesian Optimization}
	
\end{frame}

\section{Probleme und offene Fragen}

\begin{frame}[t]{Learning Rate und N\_Estimators}
	
\end{frame}

\begin{frame}[t]{NaN-Werte in den Labeln
	
\end{frame}

\begin{frame}[t]{
	
\end{frame}

\section{Pläne für die Zukunft}

\end{document}
